\documentclass[12pt,aspectratio=169]{beamer}
\usepackage[utf8]{inputenc}
\usepackage{ccfonts,eulervm,times}
\usepackage[portuges]{babel}
\usepackage{pdfpages}
\usepackage{ifpdf,ifxetex}
\usepackage{multimedia}
\usepackage{amsmath}
\usepackage{amsfonts}
\usepackage{amssymb}
\usepackage{mathptmx}
\usefonttheme{serif}
\usepackage{euler}
\usepackage{graphicx}
\usepackage{xcolor}
 \usepackage{subfigure}
\graphicspath{{Figuras/}}
\usepackage[skins,theorems]{tcolorbox}
%=========================================================================================
% Pacote para a definição de novas cores
\usepackage{xcolor}
% Definindo novas cores
\definecolor{verde}{rgb}{0.25,0.5,0.35}
\definecolor{jpurple}{rgb}{0.5,0,0.35}
\definecolor{darkgreen}{rgb}{0.0, 0.2, 0.13}
%\definecolor{oldmauve}{rgb}{0.4, 0.19, 0.28}
% Configurando layout para mostrar codigos Java
\usepackage{listings}
\newcommand{\estiloR}{
  \lstset{ %
    language=R,                     % the language of the code
    basicstyle=\footnotesize,       % the size of the fonts that are used for the code
    numbers=left,                   % where to put the line-numbers
    numberstyle=\tiny\color{gray},  % the style that is used for the line-numbers
    stepnumber=1,                   % the step between two line-numbers. If it's 1, each line
                                    % will be numbered
    numbersep=5pt,                  % how far the line-numbers are from the code
    backgroundcolor=\color{white},  % choose the background color. You must add \usepackage{color}
    showspaces=false,               % show spaces adding particular underscores
    showstringspaces=false,         % underline spaces within strings
    showtabs=false,                 % show tabs within strings adding particular underscores
    frame=single,                   % adds a frame around the code
    rulecolor=\color{black},        % if not set, the frame-color may be changed on line-breaks within not-black text (e.g. commens (green here))
    tabsize=2,                      % sets default tabsize to 2 spaces
    captionpos=b,                   % sets the caption-position to bottom
    breaklines=true,                % sets automatic line breaking
    breakatwhitespace=false,        % sets if automatic breaks should only happen at whitespace
    title=\lstname,                 % show the filename of files included with \lstinputlisting;
                                    % also try caption instead of title
    keywordstyle=\color{blue},      % keyword style
    commentstyle=\color{darkgreen},   % comment style
    stringstyle=\color{red},      % string literal style
    escapeinside={\%*}{*)},         % if you want to add a comment within your code
    morekeywords={*,...}          % if you want to add more keywords to the set
}}
%=========================================================================================
\tcbuselibrary{most}
\usepackage{cancel}
\usepackage{csquotes}
\renewcommand{\CancelColor}{\color{red}}
\usecolortheme{whale}
\usepackage{pstricks,tikz,pst-node,pst-plot,pst-eucl,pst-func}
\usepackage[all,cmtip]{xy}
\usetikzlibrary{fit, shapes.geometric, arrows}
\title{Aula  -- Introdução ao R}
\subtitle{Estatística Computacional}
%\subtitle{Matrizes}
\logo{\includegraphics[scale=0.06]{Figuras/logo_ifce.png}}
\author[Moésio]{Moésio M. de Sales}
\institute[IFCE]{Instituto Federal de Educação, Ciência e Tecnologia}
\date{\today}
%=========================================================================================
\usepackage[url,doi,style=numeric,backend=biber]{biblatex}
\addbibresource{biblio.bib}
%==========================================================================================
\useoutertheme{infolines} %tema externo
%\useoutertheme{tree} %tema externo

\useinnertheme[shadow=true]{rounded} %tema interno
\useinnertheme{circles}

\definecolor{ifce}{rgb}{1,0.7,0.5} %{vermelho,verde,azul}
\usecolortheme{spruce}
\setbeamercolor{normal text}{fg=black, bg=white}
\setbeamercolor{title}{fg=black!90, bg=green!20}
\setbeamercolor{author}{fg=red!120}
\setbeamercolor{institute}{fg=red!90!black!80!}
\setbeamercolor{date}{fg=green!20!black!90!}
\setbeamercolor{frametitle}{bg=green!20}
\setbeamercolor{structure}{fg=blue!90!black!90!, bg=green!20} %titulo do bloco e da bibliografia
\setbeamercolor{block title}{bg=green!40!white!90!} % titulo do bloco
%\setbeamercolor{block body}{bg=red!20} % titulo do bloco
%\setbeamercolor{math text}{bg=red!20} % todos os textos matem
\setbeamercolor{math text inlined}{fg=red!90!blue!90!} % texto mat em linha $$ 
\setbeamercolor{math text displayed}{fg=black!110} % texto mat em \[\]
\setbeamercolor{normal text in math text}{fg=blue!80} % texto dentro de texto mat 
%\setbeamercolor{section in head/foot}{fg=white!90} % titulo da section
\setbeamercolor{section in head/foot}{fg=white!90, bg=green!40!black!90!} % titulo da section
\setbeamercolor{subsection in head/foot}{fg=white!90} % titulo da subsection
%\setbeamercolor{subsection in head/foot}{fg=white!90, bg=red!85!blue!90!} % titulo da subsection
%paletas de cores primarias
%\setbeamercolor{palette primary}{fg=black, bg=ifce}
%\setbeamercolor{palette secondary}{fg=black, bg=green}
\setbeamercolor{palette tertiary}{fg=white}
%\setbeamercolor{palette quatenary}{fg=red, bg=red}
%modificar tamanho fonte
%\setbeamerfont{frametitle}{series=\bfseries, family=\rmfamily, size={\fontsize{15pt}{1em}}}

\usefonttheme{professionalfonts}%serif}
\setbeamercovered{transparent}
 %}
%Tamanho do block
\addtobeamertemplate{block begin}{%
  \setlength{\textwidth}{0.9\textwidth}%
}{}
\newenvironment<>{varblock}[2][.9\textwidth]{%
  \setlength{\textwidth}{#1}
  \begin{actionenv}#3%
    \def\insertblocktitle{#2}%
    \par%
    \usebeamertemplate{block begin}}
  {\par%
    \usebeamertemplate{block end}%
  \end{actionenv}}

%
\begin{document}
%=================================================
\begin{frame}<handout:0>
  \titlepage
\end{frame}
%=================================================
\section{Introdução}
%=================================================
\begin{frame}[fragile]
		\frametitle{Tipos de Estruturas\cite{crespo2017est,crawley2007,r}}
		\framesubtitle{Vetores ($vector$)}
Os vetores ($vector$) são objetos que armazenam um ou mais elementos, do tipo numeric, character ou logical.\pause

\end{frame}
%=================================================
\begin{frame}[fragile]
		\frametitle{Tipos de Estruturas}
		\framesubtitle{Indexação de Vetores}
Como um vetor só tem uma dimensão, para acessar um determinado valor do vetor é preciso fornecendo a posição do elemento no vetor entre colchetes ([]).\pause
\end{frame}
%=================================================
\begin{frame}[fragile]
		\frametitle{Tipos de Estruturas}
		\framesubtitle{Fatores ($factor$)}
		\begin{itemize}
				\item Serve para designar categorias para um vetor.\pause
				\item Essa classe é semelhante a um vetor da classe character;
		\end{itemize}

\end{frame}
%=================================================
\begin{frame}[fragile]
		\frametitle{Tipos de Estruturas}
		\framesubtitle{Indexação de Fatores}
		A indexação dos fatores funciona exatamente da mesma maneira que os vetores.
\end{frame}
%=================================================
\begin{frame}[fragile]
		\frametitle{Tipos de Estruturas}
		\framesubtitle{Tabelas de dados ($data.frame$)}
		\begin{itemize}
				\item São tabelas que armazenam um ou mais vetores de dados;\pause
				\item Possuem duas dimensões, linhas e colunas;\pause
				\item Cada coluna, será um vetor;\pause
				\item Armazenar vetores de diferentes tipos (numeric, character ou logical);\pause
				\item O comprimento de todos vetores de um data.frame deve ser igual.
		\end{itemize}
\end{frame}
%=================================================
\begin{frame}[fragile]
		\frametitle{Tipos de Estruturas}
		\framesubtitle{Matrizes ($matrix$)}
		\begin{itemize}
				\item São praticamente iguais aos data.frames;\pause 
				\item Usa-se matrix;\pause
				\item Combinação de vetores com as função cbind e rbind;
		\end{itemize}
\end{frame}
%=================================================
\begin{frame}[fragile]
		\frametitle{Tipos de Estruturas}
		\framesubtitle{Listas ($list$)}
		Listas são objetos que podem armazenam quaisquer outros tipos de objetos, são pertencentes as classe list.\pause
\end{frame}
%=================================================
\begin{frame}[fragile]
		\frametitle{Entrada de Dados}
		\framesubtitle{Dados}
Existem várias maneiras de entrar com dados no $R$.\pause
\begin{itemize}
		\item Utilizando planilhas de dados no formato .csv \pause
		\item  separado por tabulações .txt 
\end{itemize}

\end{frame}
%=================================================
\begin{frame}[fragile]
		\frametitle{Entrada de Dados}
		\framesubtitle{Descrição dos Dados}
		\[read.csv("file.csv",)\]

\end{frame}
%=================================================
\begin{frame}[fragile]
		\frametitle{Entrada de Dados}
		\framesubtitle{Dados}
Existem várias maneiras de entrar com dados no $R$.\pause
\begin{itemize}
		\item Utilizando planilhas de dados no formato .csv \pause
		\item  separado por tabulações .txt 
\end{itemize}

\end{frame}
%=================================================
%=================================================
\section{Bibliografia}
\begin{frame}
\frametitle{Referências}
%=================================================
\printbibliography
%================================================
\end{frame}

\end{document}
